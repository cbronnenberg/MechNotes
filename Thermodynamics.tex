%==============================================================================
%                                *************                                 
%                                *  PREAMBLE *                                 
%                                *************                                 
%==============================================================================


\documentclass[11pt,letterpaper,titlepage]{article}
%
% Activate the next two lines to run syntax-only checks (no dvi)
%
%...\usepackage{syntonly}
%...\syntaxonly
\usepackage{euler}
\usepackage[utf8]{inputenc}
%\usepackage{ieeetr}
%\usepackage[round]{natbib}

%...\usepackage{textcomp}
\usepackage{bbding}
% AMS packages
\usepackage{amsmath}
\usepackage{amsfonts}
\usepackage{amssymb}
%
%   Title page info
%
\author{Chris Bronnenberg}
\title{Thermodynamics Summary}
%
%  Index
%
\usepackage{makeidx}
%...\usepackage{hyperref}
\makeindex

%
% Table of Contents
%
\setcounter{tocdepth}{4}

%==============================================================================
%                            *********************
%                            *  CUSTOM EQUATIONS *                                    
%                            *********************
%    * nenewcommand's go here
%
%==============================================================================

\newcommand{\nvector}[2]{#1_1, #1_2,\ldots #1_#2}

%
% Solid Mechanics - Strain & Compatability Eq
%
\newcommand{\MilesEq}{3\times\sqrt{\frac{\pi}{2}\frac{G_{in} W_{in} Q}{f_n^2}}}
\newcommand{\StrainX}{\frac{\partial u}{\partial x}}
\newcommand{\StrainY}{\frac{\partial v}{\partial y}}
\newcommand{\StrainZ}{\frac{\partial w}{\partial z}}
\newcommand{\ShearXY}{
	\frac{1}{2}\left(\frac{\partial u}{\partial y} +  
	\frac{\partial v}
	{\partial x} \right) 
}

%
% Solid Mechanics - Strain Energy
%
\newcommand{\StrainEnergyBending}{
U_M (L)= \int_0^L \frac{M(x)^2}{2 E I(x)} dx
}
%

%
% Electricity
%
\newcommand{\KirchoffCurrentLaw}{
		\sum_{k=1}^{n} I_k = 0 \;
}
%==============================================================================
%                              ****************** 
%                              * BEGIN DOCUMENT *                              
%                              ******************
%==============================================================================
\begin{document}
\maketitle
\tableofcontents
\newpage

\section{Introduction}
%\fussy

% The following line allows Latex to hyphenate words with slashes (/) and
% avoid overfull box errors.
read\slash write

Density\index{density}, $\rho$ is a function of temperature.
$$
Q1 = U-T dS \qquad P v = n R T
$$
$$
\frac{P_2}{P_1} = \left( \frac{T_2}{T_1}\right)^{\frac{\gamma}{\gamma - 1}}
$$
Here is a sequence $\left(\nvector{a}{4}\right)$ from \cite{b_murri_payload_1987}.


Miles equation:
$$
\MilesEq
$$

Normal Strains:

\begin{align}\label{strain}
e_x  & = \StrainX \notag \\[6pt]
e_y  & = \StrainY \\[6pt]
e_z  & = \StrainZ \notag
\end{align}

It's $-30\,^{\circ}\mathrm{C}$. I will soon start to super-conduct.
The basic small strain formulation is given in \ref{strain}.

Shear Strains:
$$
\gamma_{xy} = \ShearXY
$$


Bending strain energy:
$$
\StrainEnergyBending
$$


Shock Response Spectrum \cite{LaLanne_Shock}
$$
z(t)=- \int_0^t    e^{-\xi \omega_n \tau}    \sin{ \frac{\omega_d \tau}{\omega_d} }  u''(t - \tau)    \mathrm{d} \tau
$$
What we need to know about zero shift in pyroshock testing.
\\
Zero drift can occur when high accelerations force the accelerometer's crystal to operate in the nonlinear region.
Kirchoff's current law:
$$
\KirchoffCurrentLaw
$$

$$
x \overrightarrow{e_1} + a \cos{\theta}  \overrightarrow{e_2} + a \sin{\theta}  \overrightarrow{e_3} 
$$
%
$$
\frac{\partial \psi }{\partial t}
$$

$\underset{0}{\overset{t}{\int }}f\left(t\right)dt$
Green's Functions \\
Adjoint Operators \\
$$
u L v = vL^{\star} u
$$

\begin{tabular}{cc}
	\hline 
	a & b \\ 

	c & d \\ 
	\hline 
\end{tabular} 


\[ y=x^2 \]
\bibliographystyle{plain}
\bibliography{bibfile}

\addcontentsline{toc}{chapter}{\indexname}
\printindex


\end{document}
